\documentclass[11pt]{article}
\usepackage[dvipsnames]{xcolor}
\usepackage{fancyhdr}
\usepackage{url}
\usepackage[most]{tcolorbox}
\usepackage{multirow}

\usetikzlibrary{matrix, arrows.meta, calc, positioning}
\tikzset{myarrow/.style={-Latex, rounded corners},}

\definecolor{vert}{RGB}{0,181,0}
\definecolor{oran}{RGB}{223,74,0}
\definecolor{viol}{RGB}{134,0,175}
\definecolor{roug}{RGB}{215,15,0}
\definecolor{bb}{RGB}{0,0,0}

\newtcolorbox[auto counter]{bbox}[2][]{%
    colback=white,
    colframe=bb,
    colbacktitle=white!90!roug,
    coltitle=black,
    fonttitle=\bfseries, 
    enhanced,
    attach boxed title to top left={yshift=-2mm, xshift=0.5cm},%
    #1% For possible options
}
\topmargin=-5mm
\evensidemargin=0cm
\oddsidemargin=0cm
\textwidth=16cm
\textheight=22cm
\addtolength{\headheight}{1.6pt}
\newcommand{\cancel}[1]{}
\newcommand{\mymark}{$^*$}
\setlength\parindent{24pt}

\newcommand{\lastupdate}{August 2015}

\lhead{\sc IACR Policy for Cryptology Schools}
\rhead{\sc \lastupdate}

\title{\bf IACR Policy for Cryptology Schools}
\author{\mbox{}}

\date{\lastupdate
 \footnote{The most recent version of this document
    can be obtained from \protect\url{http://www.iacr.org/docs/}.\newline
  Editors of this document: M. Abdalla, A. Boldyreva, C. Cachin, A. Kiayias, B. Warinschi (2014).}}

\newcommand{\msf}[1]{\ensuremath{{\mathsf {#1}}}}
\newcommand{\f}[1]{\ensuremath{\mathcal{#1}}}
\newcommand{\F}{\f{F}}
\newcommand{\C}{\mathcal{C}}
\newcommand{\con}[1]{\msf{Contract_{#1}}}
%\newcommand{\Fsync}[2]{\ensuremath{\F_{\msf{sync},#1,#2}}}
\newcommand{\Fsync}[2]{\ensuremath{\F_{\msf{BD-SEC}}(#1,#2)}}
\newcommand{\Fstate}{\ensuremath{\F_{\msf{state}}}}
\newcommand{\Fpay}{\ensuremath{\F_{\msf{pay}}}}
\newcommand{\Gledger}{\ensuremath{\f{G}_{\msf{ledger}}}}
\newcommand{\globalf}[1]{\ensuremath{{\overline{\mathcal{#1}}}}}
\newcommand{\todo}[1]{\textcolor{red}{#1}}
\newcommand{\edict}{\{\}}
\newcommand{\lar}{\leftarrow}
\newcommand{\rar}{\rightarrow}
\newcommand{\Init}{{\bf \color{NavyBlue} Init}~}
\newcommand{\OnInput}{{\bf \color{NavyBlue} On input}~}
\newcommand{\Allinputs}{{\bf \color{Cerulean} All other input~}}
\newcommand{\OnAdvInput}{{\bf \color{BrickRed} On input}~}

\makeatletter
\newcommand{\inmsg}[1]{%
	(#1\checknextarg}
\newcommand{\checknextarg}{\@ifnextchar\bgroup{\gobblenextarg}{)~}}
\newcommand{\gobblenextarg}[1]{, #1\@ifnextchar\bgroup{\gobblenextarg}{)~}}
\makeatother

\newcommand{\transfermsg}{\inmsg{transfer}{to}{val}{data}{from}}
\newcommand{\createmsg}{\inmsg{contract \ create}{addr}{val}{data}{private}{from}}
\newcommand{\reject}{\textbf{reject}~}
\newcommand{\ignore}{\textbf{ignore}~}
\newcommand{\For}{\textbf{For}~}
\newcommand{\Env}{\ensuremath{\mathcal{Z}}}
\newcommand{\While}{\textbf{While}~}
\newcommand{\Buffer}{\textbf{Buffer}~}
\newcommand{\Send}{\textbf{Send}~}
\newcommand{\Output}{\textbf{Output}~}
\newcommand{\Leak}{\textbf{Leak}~}
\newcommand{\In}{\textbf{in}~}
\newcommand{\If}{\textbf{If}~}
\newcommand{\Else}{\textbf{Else}~}
\newcommand{\Return}{\textbf{Return}~}
\newcommand{\pluseq}{\ensuremath{\mathrel{+}=}}
\newcommand{\Adv}{\ensuremath{\mathcal{A}}}
\newcommand{\Partyi}{\ensuremath{\mathbf{P_i=(sid,pid)}}}
%\newcommand{\qquad}{\quad \quad}
\newcommand{\qqquad}{\qquad \quad}
\newcommand{\qqqquad}{\qqquad \quad}
\newcommand{\qqqqquad}{\qqqquad \quad}


\begin{document}

\pagestyle{fancy}
\pagenumbering{arabic}

\maketitle

\section{Related Works}
Nodes on \cite{kiayias2016fair}:

\begin{itemize}
    \item This model starts with describing how to model execution of \emph{synchronous} protocols that can access a global setup clock.
    \item In a previous treatment, the clock in UC was local to each party and it would have to receive update messages from the other parties (everyone is doing this operation). Hence, with GUC the environment can control the clock speed and define when clock updates happen (as other protocol sessions might also be accessing it).
\end{itemize}

There are several works from the past few years that try to model a blockchain within the Universal Composability framework---some attempting to model it in its extendion, (G)UC \cite{uc, guc}.

% Modelling blockchain for reward/penalty in a fair MPC setting, downfalls reported in next paragraph
Kiayias et al.~\cite{kiayias2016fair} models a Bitcoin-like blockchain for fair and robust multi-party computation.
It is motivated by the impossibility result for fairness in secure MPC \footnote{Fairness in MPC is defined as: either all parties learn the output or none of them do.} and circumeventing it by imposing monetary penalties on participants.
The model consists of two global functionalities, $\globalf{G}_{\msf{clock}}$ and $\globalf{G}_{\msf{blockchain}}$.
The blockchain functionality enables the expected functionality like submitting tranasctions, validating them, batching them into blocks, and allowing an adversary to reorder transcations.
Because of the GUC framework, the state of the blockchain is available to all parties including the environment and any other protocol sessions (or dummy parties).
This work however, fails to prove that their model of the blockchain is GUC-realized in any currently existing blockchain system.
Such a security proof is essential as it provide credibility to the possibility of implementing protocols in the $\globalf{G}_\msf{blockchain}$-hybrid world.
Furthermore, the assumptions that are made for the blockchain and what the adversary can do severly limit the scope of adversaries in the rearl-world.
The first failure of this model is to consider an adversary which can change the view some parties have of the blockchain state.
For example, if the adversary mines a new block and keeps it a secret, or if some nodes have not received new blocks because of communication delays.
Another failure is that all transactions in the buffer between blocks are always included in the next block.
This, again, prevents a miner-like adversary which can censor transactions and delay their entry into the chain.
Finally, the state of the blockchain is updated at fixed time intervals which does not accurately convey the consensus model of Bitcoin or Ethereum.

% Bitcoin composable treatment in GUC
Badertscher et al.~\cite{badertscher2017bitcoin} attempt to solve these problems by allowing a more unrestricted in the GUC framework.
The shared functionality in this case is a global clock functionality, $\globalf{G}_\msf{clock}$, which enables modelling a synchronous system in the UC framework by proceeding in rounds.
Because it is a shared functionality, the clock allows any other protocol session in the environment to be synchronized with the challenge protocol. 
The blockchain functionality is a local functionality (only available to the parties within the protocol session) that allows the adversary to have more power in what it can do.
The adversary can inject transactions and modify the state of the chain that all parties that query it can see.
This is accomplished by allowing a maximum distance, $d$, that the adversary can specify and return a prefix of the chain which is at most a distance $d$ from the head of the chain.
Furthermore, the adversary can choose exactly which transactions are allowed to be in the next block.
The blockchain functionality is modularized by allowing the definition of subroutines that capture extending the blockchain state (specifically for Bitcoin in this paper).
The authors of this work admit that the paper's only intent is to model the Bitcoin blockchain hence the choice to use the ledger as only a local functionality. 
This prevents other protocol sessions from using the same blockchain (definitely a limitation of modelling the reality of a blockchain environment).
Furthermore, this paper makes the argument that it is dangerous to have a global ledger functionality as such replacement does not ``in general, preserve a realization proof of some ideal functionality $\F{}$ that is conducted in a ledger-hybrid world, because the simulator in that proof might rely on specific capabilities that are not available any more after the replacement (as the global setup is also replaced in the real world)''.
It claims that~\cite{canetti2016universally} provides a sufficient condition for such a replacement, but that the condition is too strong to be satisfied by any ledger implementation.

% UC with a global PKI
\begin{itemize}
	\item The leading example in this paper is that the standard PKI that has been described in previous works does not work in the generalized case in most attempts.
			When it does have GUC, it is often required to also have this non-transferability (deniability) property where signatures received by a party from another can not be re-used to prove the authenticity of a message.
			This means that signatures are only valid and checkable by the intended recipient. 
			In the real world this requirement is bogus because signatures are re-usable and general EU-CMA signatures behave in this way meaning that the functionalities can not be realized by a useful signature scheme.
			This is why they propose a new composition theorem which allows swapping of shared functionalities with EUC-simulated other functionalities with weaker properties (``verify this choice of words'').
	\item One lacking concept in tihs paper is that a particular ITI may only register a single key with the Cert and Bulliten Board functionalities.
			The claim made is that it is possible to realize $\F_{\msf{cert\_auth}}$, but not by going through certificates.
\end{itemize}




\section{Preliminaries}
This section will briefly outline the UC model and the synchronous model propose by Katz et al.~\cite{katz-clock}.
Next, we introduce the asynchronous byzantine broadcast protocol proposed in ~\cite{bracha-broadcast} and how it is cast to the synchronous model.

\subsection{UC Framework}
\todo{brief on the UC framework. Key points are adversary model, existence of environment, real/ideal paradigm, composition operation, asynchronous communication}

\subsection{Synchronous Protocols in UC}
As mention above, the UC framework is built to model asynchronous communication, i.e. there are not guarantees on the delay of messages from one party to another.
%More importantly, this means that it is impossible for any party to distinguish between a faulty process and one whose messages haven't been received yet.
More importantly this makes it difficult to reason about liveness and termination properties of protocols as faulty processes are indistinguishable from those whose messages haven't arrived yet.
Naturally, the asynchronous model stronger than a synchronous model as less assumptions are required when proving the security of protocols.
However, in some cases additional properties are required such as guaranteed termination and imput completeness.
In the case of distributed applications, we wish to reason about their liveness properties which are difficult to achieve in asynchronous systems~\cite{flp}.

\paragraph{Model} 
The first step in the synchronous model is to apply bounded delay to the point-to-point channels that parties have between each other.
One of the key reasons termination can not be guarnateed in the asynchronous model is that the adversary can impose ubounded delay and force processes to wait indefinitely and not make any progress.
We use the bounded delay channel proposed in ~\cite{katz-clock}, \Fbdsec, described in Figure~\ref{fig:bdsec}.

\begin{figure}
	\begin{bbox}[title={\textbf{Functionality} $\F_{\msf{BD-SEC}}^{\delta,\ell}(p_s,p_r)$}]

Initialize $M := \bot$ and $D := 1$ and $\hat{D} := 1$

-- \OnInput \inmsg{m} from $\mathbf{p_s}$:

	\dquad Set $D := 1$, $M := m$

	\dquad \Leak ({\bf send},$M$) $\rightarrow \mathcal{A}$

-- \OnInput \inmsg{fetch} from $p_r$:

	\dquad Set $D = D - 1$

	\dquad \If $D = 0$: \Send ({\bf sent},$M$) $\rightarrow p_r$

-- \OnInput \inmsg{delay}{T} from $\mathcal{A}$:

	\qquad \If $\hat{D} + T \leq \delta$: 

		\qquad \quad $D := D + T$, $\hat{D} := \hat{D} + T$

		\qquad \quad \Send ({\bf delay},$T$) $\rightarrow \mathcal{A}$

-- \OnInput \inmsg{replace}{$m'$}{$T'$} from $\mathcal{A}$:

	\qquad \If $p_s$ corrupted, $D > 0$ and $T'$ is valid:

		\qquad \quad $D := T'$ and $M' = m$
\end{bbox}


	\caption{A bound delay channel betwee two parties. Parameterized by bounded delay, $\delta$, and leak function, $\ell$. Without any other crypgoraphy, the most basic leak function is $\ell(m) := m$.}
	\label{fig:bdsec}
\end{figure}

The channel forces the receiver to continuously poll the the channel until a message is available.
An internal count $D$ tracks how many activations are remaining until the message it delivered.
The adversary can delay the message at any time but the total delay imposed, denoted by $\hat{D}$, is bounded by $\delta$.
Howeer, a bounded delay channel alone is not sufficient to achieve input completeness and guaranteed termination.
Without another functionality to synchronize all participants, some party may begin computation and sending message for the next round before other honest parties have completed the current round.
\todo{Add some more exposition about clock functionality}.

The synchronization primitive in ~\cite{katz-clock}, denotes \Fclock, is shown in Figure ~\ref{fig:functionality:clock}.
The functionality maintains an internal but $d_i$ for each party in $\mathcal{P}$. 
In every round, each party performs all necessary actions and indictes it is finished by sending the \msf{RoundOK} signal to \Fclock.
As mentioned previously, the party waits until its bit is reset, i.e. every party has finished its round.
\todo{the fclock functionality in katz only waits until every corrupted party indicates the round is over, is that correct or a type? why (not)?}

\begin{figure}
	\begin{bbox}[title={\textbf{Functionality} $\F_{\msf{clock}} (\mathcal{P})$}]

Intialize $\forall p_i \in \mathcal{P}: d_i := 0$

\vspace{2mm} \hrule \vspace{2mm}

-- \OnInput \inmsg{RoundOK} from \Partyi:

	\dquad $d_i = 1$

	\dquad \If $\forall p_i \in \mathcal{P}: d_i = 1$:

	\dquad \quad $d_i := 0$ for all $p_i \in \mathcal{P}$

	\dquad \Leak $(\msf{switch},\mathbf{P_i}) \rightarrow \mathcal{A}$

-- \OnInput \inmsg{RequestRound} from \Partyi: 

	\dquad \Send $d_i \rightarrow \mathbf{P_i}$

-- \OnInput \inmsg{corrupt}{$p_i$} from $\mathcal{A}$:

	\dquad Set $\mathcal{H} := \mathcal{H} \cup \{p_i\}$

\end{bbox}


	\label{fig:functionality:clock}
\end{figure}

\paragraph{Protocols in the $\{\F_{\msf{BD-SEC}}^{\delta,\ell}, \F_{\msf{clock}}\}$-hybrid model} 
We briefly recap how a protocol operates in this world to achieve the desired properties.
Each protocol begins with its local round set to 1.
In each round the protocol performs some local computation, receives messages from other parties from previous rounds, and potentially sends messages of its own or future rounds from its incoming and outgoing channels, \Fbdsec.
The last thing each itms does in the round is send the \msf{RoundOK} signal to $\F_{\msf{clock}}$.
If it is activated again, it checkes whether the all other parties have also indicated completion of their round.
When it's bit is reset to 0, it increments its local round counter and begins executing the next round.
Otherwise, it doesn't do anything until the round is finished.


\begin{figure}[!ht]
	
\begin{bbox}[title=\msf{ExecTx(to, val, data, from)}]

$\msf{nonces[from]} \lar \msf{nonces[from]} + 1$

\If $\msf{balances[from]} < \msf{val}$: \reject

$\msf{balances[from]} \lar \msf{balances[from]} - \msf{val}$

$\msf{balances[to]} \lar \msf{balances[to]} + \msf{val}$

$\msf{receipts[from,nonces[from]]} \lar \msf{CreateTxRef(val, from)}$

\If $\msf{to} \in \msf{contracts}$:

	\quad $ret \lar \msf{Exec(to, val, data, from)}$

	\quad $\msf{txs[from, nonces[from]]} \lar ret$

\end{bbox}

	\label{fig:functionality:exectx}
\end{figure}

\begin{figure}[!ht]
\begin{bbox}[title=\msf{ExecContractCreate(addr, val, data, from, private)}]

$\msf{nonces[from]} \lar \msf{nonces[from]} + 1$

\If $\msf{balances[from]} < \msf{val}$: \reject

$\msf{balances[from]} \lar \msf{balances[from]} - \msf{val}$

$\msf{balances[to]} \lar \msf{balances[to]} + \msf{val}$

$(\msf{functions}, \msf{args}) := \msf{data}$

$r \lar \msf{functions.init}(args)$

$\msf{contracts[addr]} = \msf{functions}$

$\msf{restricted[addr]} = \msf{private}$

\If $\neg r$:

\quad $\msf{balances[from]} \lar \msf{balances[from]} + \msf{val}$

\quad $\msf{balanaces[to]} \lar \msf{balances[to]} - \msf{val}$
\end{bbox}

\label{fig:functionality:execcreate}
\end{figure}

\begin{figure}[!ht]
\begin{bbox}[title=$\globalf{G}_{\msf{ledger}}$]

Initialize $\msf{txqueue} := \edict$, $\msf{contracts} := \edict$, $\msf{newtxs} := \edict$, $\msf{nonces} := \edict$ \msf{balances} := \edict, $\Delta := 8$, $rnd := 0$\\

\OnInput \transfermsg from \Partyi:

	\quad If $\msf{balances[fro]} < \msf{val}$: {\bf reject}

	\quad $\msf{nonces[from]} \lar \msf{nonces[from]} + 1$

	\quad $\msf{newtxs[from,nonces[from]} \lar \transfermsg$
	
	\quad {\bf leak} \transfermsg to \Adv

\OnInput \createmsg from \Partyi:

	\quad If $\msf{balances[from]} < \msf{val}$: {\bf reject}
	
	\quad $\msf{nonces[from]} \lar \msf{nonces[from]} + 1$

	\quad $caddr \lar \msf{ComputeAddr}(from)$
	
	\quad If $caddr \neq addr$: \reject

	\quad If \msf{len(data)} = 0: \reject

	\quad $\msf{newtxs[from,nonces[from]} \lar \transfermsg$

	\quad {\bf leak} \createmsg to \Adv

\OnInput \inmsg{tick}{addr} from \Partyi:

	\quad $rnd += 1$

	%\quad $\msf{balances[sid,pid]} \pluseq 1000000$
	\quad $\msf{balances[addr]} \pluseq 1000000$

	\quad \For \msf{tx} \In \msf{txqueue[rnd]}: 

		\qquad \If $tx[0] = \msf{transfer}$:
			
			\qqquad $(\msf{transfer, to, val, data, from}) \lar tx$

			\qqquad \msf{ExecTx(to, val, data, from)}

		\qquad If $tx[0] = \msf{contractcreate}$:

			\qqquad $(\msf{contractcreate, addr, val, data, private, from}) \lar tx$
	
			\qqquad \msf{ExecContractCreate(addr, val, data, private, from)}

\hrulefill

\OnAdvInput \inmsg{delayTx}{from}{nonce}{rounds} from \Adv:

	\quad $tx \lar \msf{newtxs[from,nonce]}$

	\quad Add $tx$ to \msf{txqueue[rnd + rounds]}

	\quad Remove $tx$ from \msf{newtxs}

\OnAdvInput \inmsg{tick}{addr}{permutation} from \Adv:
	
	\quad Apply $permutation$ to \msf{txqueue[rnd]}

	\quad Run honest party mining with \msf{addr}

\end{bbox}

	\caption{Ideal functionality representing a basic ledger with adversarial methods for delaying/reordering transactions and smart contract support}
	\label{fig:functionality:ledger}
\end{figure}

\begin{figure}
	\begin{bbox}[title=Protection Wrapper $\f{W}_p$]

\OnInput \inmsg{transfer}{to}{val}{data}{from} from \Partyi:

	\quad $to \lar$

\end{bbox}

	\caption{Protection wrapper for the ledger to maintain indistinguishability.}
\end{figure}

\begin{figure}
	\begin{bbox}[title=$U_{pay}$]

$U_{pay} (\msf{state}, (\msf{input_L},\msf{input_R}), \msf{aux}_{in})$:

\quad \If $\msf{state} = \bot$: $\msf{state} := (0,\emptyset,0,\emptyset)$

\quad parse \msf{state} as $(\msf{cred_L},\msf{oldarr_L},\msf{cred_R},\msf{oldarr_R})$

\quad parse $\msf{aux}_{in}$ as $\{ \msf{deposits}_i \}_{i \in \{L,R\}}$

\quad \For $i \in \{L,R\}$:

	\qquad \If $\msf{input}_i = \bot$: $\msf{input}_i := (\emptyset,0)$

	\qquad parse $\msf{input}_i$ as $\msf{arr}_i,\msf{wd}_i$

	\qquad $\msf{pay}_i := 0, \msf{newarr}_i := \emptyset$

	\qquad \While $\msf{arr}_i \neq \emptyset$:

		\qqquad $e \leftarrow \msf{pop}(\msf{arr}_i)$

		\qqquad \If $e + \msf{pay}_i \leq \msf{deposits}_i + \msf{cred}_i$:

			\qqqquad $\msf{newarr}_{\neg i} \leftarrow e$

			\qqquad $\msf{pay}_i += e$

	\qquad \If $\msf{wd}_i > \msf{deposits}_i + \msf{cred}_i - \msf{pay}_i: \msf{wd}_i := 0$

\quad $\msf{cred_L} += \msf{pay_R} - \msf{pay_L} - \msf{wd_L}$

\quad $\msf{cred_R} += \msf{pay_L} - \msf{pay_R} - \msf{wd_R}$

\quad \If $\msf{wd_L} \neq 0$ or $\msf{wd_R} \neq 0$:

	\qquad $\msf{aux}_{out} := (\msf{wd_L},\msf{wd_R})$

\quad \Else: $\msf{aux}_{out} := \bot$

\quad $\msf{state} := (\msf{cred_L},\msf{newarr_L},\msf{cred_R},\msf{newarr_R})$

\quad \Return $(\msf{aux}_{out}, \msf{state})$

\end{bbox}

	\caption{Update function for a payment channel. Given as a parameter to \Fstate. It defines the format of the \msf{state} and its updates.}
\end{figure}

\begin{figure}
	\begin{bbox}[title=$\Pi_{pay}$: $\msf{Contract_{pay}}$]

\Init $(\msf{P_L}, \msf{P_R})$:

\quad $\msf{deposits}_L, \msf{deposits}_R := 0$

% deposit($X)
\OnInput \inmsg{deposit}(tx) from \Partyi:

	\quad $\msf{deposits}_i += \msf{tx.value}$

	\quad $\msf{out}(\msf{deposits}_L, \msf{deposits}_R)$

% aux_out 
\OnInput \inmsg{output}{\msf{aux_{out}}}{\msf{tx}}:

	\quad parse $\msf{aux_{out}}$ as $(\msf{wd_L},\msf{wd_R})$

	\quad \For $i \in \{L,R\}$: $\msf{send}(P_i, \msf{wd_i})$

\end{bbox}

	\caption{Contract pay}
\end{figure}

\begin{figure}
	\begin{bbox}[title=$\Pi_{pay}$]

Initialize $\msf{arr_i} = \emptyset, \msf{pay_i} = 0, \msf{wd_i} = 0, \msf{paid_i} = 0$

$\msf{Contract_{pay}}$ identifier $\mathcal{C}$ 

\Send $(\emptyset, 0) \rightarrow \F_{state}$

\OnInput \inmsg{ping} from \Env:

	\quad $\Send (\msf{read}) \rightarrow \Fstate$

% New state from F_state
\OnInput $(\msf{cred_L}, \msf{new_L}, \msf{cred_R}, \msf{new_R})$ from $\Fstate$:

	\quad \For $e \in \msf{new_i}$:

		\qquad \textbf{Output} $(\msf{receive}, e)$

		\qquad $\msf{paid_i} += e$
	
	\quad \Send $(\msf{arr_i}, \msf{wd}-\msf{wdn}) \rightarrow \Fstate$

	\quad $\msf{arr_i} \leftarrow \emptyset$

	\quad $\msf{wdn_i} \leftarrow \msf{wd_i}$

% Pay($X)
\OnInput \inmsg{pay}{\$X} from \Env:

	\quad $\msf{Contract_{Pay}} \leftarrow \Gledger.\msf{contract}(\mathcal{C})$

	\quad \If $\$X \leq \msf{Contract_{Pay}}.\msf{deposits_i} + \msf{paid_i} - \msf{pay_i} - \msf{wd_i}$:

		\qquad $\msf{arr_i} \leftarrow \$X$

		\qquad $\msf{pay_i} += \$X$

% Withdraw($X)
\OnInput \inmsg{withdraw}{\$X} from \Env:

	\quad $\msf{Contract_{Pay}} \leftarrow \Gledger.\msf{contract}(\C)$

	\quad \If $\$X \leq \con{Pay}.\msf{deposits_i} + \msf{paid_i} - \msf{pay_i} - \msf{wd_i}$:

		\qquad $\sf{wd_i} += \$X$

\end{bbox}

	\caption{Local protocol for parties to follow for a payment channel between two parties. Parties can pay, deposit into, or withdraw from the channel.}
\end{figure}

\begin{figure}
	\begin{bbox}[title={$\Fstate (U, C)$}]

\end{bbox}

	\caption{The ideal functionality \Fstate. The functionality proceeds in rounds and waits for parties to provide input. When all parties have provided input or the round deadline has passed, a state update is executed. Contract output is given to \Gledger in the form of a transaction. Parties must explicitly \msf{ping} the functionality in order to make progress. }
\end{figure}

\begin{figure}
	\begin{bbox}[title={$\Fpay (P_L,P_R,\Delta)$}]

Initialize $\msf{bal_L} := 0, \msf{bal_R} := 0$

\OnInput \inmsg{pay}{\$X} from \Partyi:

	\quad \If $\msf{bal}_i < \$X$: ignore  

	\quad \Leak $(\msf{pay},P_i,\$X) \rightarrow \mathcal{A}$

	\quad $\msf{bal}_i -= \$X$

	\quad \If $P_{\neg i}$ is honest: \Buffer $((\msf{receive},\$X),1,P_{\neg i})$

	\quad \Else: \Buffer $((\msf{receive},\$X),O(\Delta),P_{\neg i})$ 

\OnInput \inmsg{withdraw}{\$X} from \Partyi:

	\quad \If $\msf{bal}_i < \$X$: ignore

	\quad \Leak $(\msf{withdraw},P_i,\$X) \rightarrow \mathcal{A}$

	\quad $\msf{bal}_i -= \$X$

	\quad \Send $(\msf{transfer}, (\mathbf{sid},\mathbf{pid}), \$X, \bot, \msf{mysidsomehow}) \rightarrow \Gledger$

\OnInput \inmsg{deliver}{msg,$P_i$}:

	\quad \If $(\msf{receive},e) = msg$:

		\qquad $\msf{bal}_i += \$X$
	

\end{bbox}

	\caption{The payment channel functionality. Unlike $\Fstate$, doesn't need any notion of rounds until it must deal with on-chain transactions for deposits. Buffering for $O(\Delta)$ rounds implies the adversary can choose the number.}
\end{figure}

\begin{figure}
	%\begin{bbox}[title={Wrapper $\mathcal{W} (\mathcal{F},\mathcal{C}_1,...,\mathcal{C}_k)$}]
%
%Initialize $\msf{outputs} := \emptyset$, $\msf{buffer} := \emptyset$
%
%\OnInput $\inmsg{buffer}{msg}{\delta}{P_i}$ from $\mathcal{F}$:
%	
%	\quad $\msf{buffer}[\Gledger.\msf{rnd}+\delta].\msf{append}(msg,P_i)$ 
%
%\OnInput \inmsg{read} from \Partyi:
%
%	\quad $\msf{out} := \msf{outputs}[P_i]$
%
%	\quad $\msf{outputs}[P_i] := \emptyset$
%
%	\quad $\Send \msf{out} \rightarrow P_i$
%
%\Allinputs m from \Partyi:
%
%	\quad \Send $m \rightarrow \F$
%
%\vspace{2mm} \hrule \vspace{2mm}
%
%When activated, do the following subroutine before processing the message:
%
%	\quad \For $(msg,P_i) \in \msf{buffer}[\Gledger.\msf{rnd}]$:
%
%		\qquad \Send $(\msf{deliver},msg,P_i) \rightarrow \F$
%
%		\qquad $\msf{outputs}[P_i].\msf{append}(msg)$
%
%\end{bbox}
\begin{bbox}[title={\textbf{Wrapper} $\mathcal{W}_{\msf{synchronous}} (\mathcal{F})$}]

Proceed in rounds starting in round $r=1$.

-- On first activation of $p_i$ in round $r$:

	\qquad \For $p_j \in \mathcal{P}$:

		\qquad \quad Get message $m$ from $\Fsync{p_j}{p_i}$

		\qquad \quad Deliver message to \F

	\todo{needs to be finalized in code first}

\end{bbox}

	\caption{The wrapper $\mathcal{W}$ that provides common function for all functionalities. In $\Fstate$ for example, the wrapper enables functionalities to buffer sending output to the parties in the protocol. When the wrapper sends a message to its functionality $\F$, it does not constitute an \msf{ITM} to \msf{ITM} write as they are both running on the same \msf{ITM}.}
\end{figure}

\begin{figure}
	\begin{bbox}[title={$\F_{\msf{bcast}} (p_L, p_1...p_n)$}]

Initialie $\msf{buffer} := \emptyset$, $\msf{lastRound} \leftarrow -1$, $\msf{round} \leftarrow 0$

\OnInput \inmsg{broadcast}{msg} from \Partyi:

	\quad \If $\Partyi \neq p_L$: ignore

	\quad \Leak $(\msf{msg},\msf{round} + 1) \rightarrow \mathcal{A}$

	\quad $\msf{buffer}[\msf{round}+1] \leftarrow \msf{msg}$


\OnInput \inmsg{deliver}{msg}{to} from $\mathcal{A}$:
	
	\quad \If $\msf{to} \notin (p_1,...,p_n)$: ignore

	\quad $m,r \leftarrow \msf{msg}$

	\quad \If $\msf{m} \in \msf{buffer}[r]$:

	\quad \quad \Send $(m) \rightarrow \msf{to}$

\vspace{2mm} \hrule \vspace{2mm}

When activated do that following first:

	\quad \Send $(\msf{clockread},) \rightarrow \globalf{G}_{\msf{clock}}$

	\quad $\msf{rnd} \leftarrow wait(\globalf{G}_{\msf{clock}})$

	\quad \If $\msf{rnd} > \msf{round}$:

	\quad \quad $\msf{lastRound} \leftarrow \msf{round}$
	
	\quad \quad $\msf{round} \leftarrow \msf{rnd}$

\end{bbox}

\end{figure}

\begin{figure}
	\begin{bbox}[title={$\globalf{G}_{\msf{clock}}$}]

Intialize $\msf{registry} := \emptyset$, $\msf{dp} := \emptyset$, $\msf{sessionT} := \emptyset$

\OnInput \inmsg{register} from \Partyi:

	\quad \If $\msf{pid} \notin \msf{registry}[\msf{sid}]$:

	\quad \quad Add $\msf{pid}$ to $\msf{registry}[\msf{sid}]$

	\quad \quad \If $\msf{sid} \notin \msf{sessionT}$:
		
	\quad \quad \quad $\msf{sessionT}[\msf{sid}] := 0$

\OnInput \inmsg{clockread} from \Partyi:

	\quad \If $\msf{sid} \notin \msf{registry}$: ignore

	\quad \Send $\msf{sessionT}[\msf{sid}] \rightarrow P_i$ 

\OnInput \inmsg{clockupdate} from \Partyi:

	\quad \If $\msf{sid} \notin \msf{registry}$: ignore

	\quad $\msf{dp}[\msf{sid},\msf{pid}] := 1$

	\quad \If $\forall \msf{p}$, $\msf{dp}[\msf{sid},\msf{p}] = 1$:

	\quad \quad $\msf{sessionT}[\msf{sid}] += 1$

\end{bbox}

\end{figure}

\begin{figure}
	\begin{bbox}[title={$\Pi_{\msf{state}} (\msf{sid}, \msf{pid}, U, \mathcal{C}_{\msf{aux}}, \mathcal{C}_{\msf{state}}, \msf{leader}, \msf{peers}=p_1,...p_n)$}]

Initialize $\msf{round} := 0$, $\msf{pinputs} := \emptyset$, $\msf{aux_in} = []$, $\msf{flag} := \msf{OK} \in \{\msf{OK},\msf{PENDING}\}$, $\msf{aux\_out} := \emptyset$, $\msf{state} := \emptyset, \msf{psigs} := \emptyset, \msf{lastRound} := -1$

$\msf{step} := input \in \{input,batch,commit\}$

\vspace{2mm} \hrule \vspace{2mm}

If $\msf{pid} = \msf{leader}$, do the following:

\OnInput \inmsg{INPUT}{$v_i$}{r} from \Partyi:

	\quad \If $r \neq \msf{round}$: \ ignore

	\quad \If first input from $P_i$ in round $r$: \ Add $v_i$ to $\msf{pinputs}$ 

	\quad \If $\forall p_i, v_i \in \msf{pinputs}$:
	
	\quad \quad \Send $(\msf{BATCH}, \msf{r}, \msf{aux_in}, \msf{pinputs}) \rightarrow \F_{\msf{bcast}}$ 


\OnInput \inmsg{SIGN}{$\sigma$}{r} from \Partyi:

	\quad \If $\msf{step} \neq commit$ or $\msf{r} \neq \msf{round}$ or $\msf{Verify}(\sigma, \msf{r}, \msf{aux\_out}, \msf{state}) \neq 1$: \ ignore

	\quad \If first sign from $P_i$ in round $r$: \ Add $(P_i,\msf{r},\sigma)$ to $\msf{psigs}$

	\quad \If $\forall p_i, (p_i,\msf{r},\_) \in \msf{psigs}$:

	\quad \quad \Send $(\msf{COMMIT}, \msf{r}, \{\sigma\}_{i}) \rightarrow \F_{\msf{bcast}}$


\vspace{2mm} \hrule \vspace{2mm}

{\bf If $\msf{flag} = \msf{OK}$}:

\OnInput \inmsg{input}{v} from $\mathcal{Z}$:

	\quad \If $\msf{step} \neq input$ or $r \neq \msf{round}$: \ ignore

	\quad $\msf{step} := batch$ 

	\quad \Send $(\msf{INPUTS}, \msf{v}, \msf{round}) \rightarrow \msf{leader}$ 

\OnInput \inmsg{\msf{BATCH}}{r}{aux\_in}{pinputs} from $\F_{\msf{bcast}}$:

	\quad \If $\msf{step} \neq batch$ or $r \neq \msf{round}$: \ ignore

	\quad $\msf{step} := commit$

	\quad todo: how to imply ``recent'' value of aux\_in??

	\quad $\msf{state},\msf{aux\_out} := U(\msf{state}, \msf{pinputs}, \msf{aux\_in}, \msf{round})$

	\quad $\sigma \leftarrow \msf{Sign}(r || \msf{aux\_out} || \msf{state})$

	\quad \Send $(\msf{SIGN}, \sigma) \rightarrow \msf{leader}$


\OnInput \inmsg{\msf{COMMIT}}{r}{$\{\sigma_r\}_i$} from $\F_{\msf{bcast}}$:

	\quad \If $\msf{r} \neq \msf{round}$ or $\msf{step} \neq commit$ or $\left( \bigvee_{\sigma_i} \msf{Verify}(\sigma_i,\msf{r},\msf{aux\_out},\msf{state}) = 0 \right)$: \ ignore

	\quad \msf{lastCommit} := $(\msf{state},\msf{aux_out},\{\sigma_r\}_i)$

	\quad \msf{lastRound} := \msf{r}

	\quad \msf{round} := \msf{lastRound}+1

	\quad $\msf{step} := input$

\end{bbox}

\end{figure}

\section{Three-Phase Commitment}

\subsection{Synchronous Bracha Broadcast}

\begin{figure}[!h]
	\begin{bbox}[title={\textbf{Functionality} $\F_{\msf{BD-SEC}}^{\delta,\ell}(p_s,p_r)$}]

Initialize $M := \bot$ and $D := 1$ and $\hat{D} := 1$

-- \OnInput \inmsg{m} from $\mathbf{p_s}$:

	\dquad Set $D := 1$, $M := m$

	\dquad \Leak ({\bf send},$M$) $\rightarrow \mathcal{A}$

-- \OnInput \inmsg{fetch} from $p_r$:

	\dquad Set $D = D - 1$

	\dquad \If $D = 0$: \Send ({\bf sent},$M$) $\rightarrow p_r$

-- \OnInput \inmsg{delay}{T} from $\mathcal{A}$:

	\qquad \If $\hat{D} + T \leq \delta$: 

		\qquad \quad $D := D + T$, $\hat{D} := \hat{D} + T$

		\qquad \quad \Send ({\bf delay},$T$) $\rightarrow \mathcal{A}$

-- \OnInput \inmsg{replace}{$m'$}{$T'$} from $\mathcal{A}$:

	\qquad \If $p_s$ corrupted, $D > 0$ and $T'$ is valid:

		\qquad \quad $D := T'$ and $M' = m$
\end{bbox}


\end{figure}
%\begin{figure}
%	%\begin{bbox}[title={Simulator $S_{\msf{Bracha}}$}]
%
%Simulate real-world parties $\overline{\mathcal{P}} = p_1,...,p_n$ and $\Fsync{p_i}{p_j}, \forall p_i,p_j \in \overline{\mathcal{P}}$
%
%Simulate instance $\overline{\F}$ of $\F_{\msf{clock}}$.
%
%Designate same dealer $\overline{\mathcal{D}}$ as environment.
%
%Simulate dummy adversray $\mathcal{A}_{\mathcal{D}}$
%
%\vspace{2mm} \hrule \vspace{2mm}
%
%Case \#1 ( Dishonest $\mathcal{D}$ ):
%
%\OnInput \inmsg{input}{v} from $\mathcal{Z}$ for $\mathcal{D}$:
%
%	\quad \Send (input,v) $\rightarrow \mathcal{A}_{\mathcal{D}}$ {\em (Passthrough for corrupted parties in real world)}
%
%\OnInput \inmsg{m} from $\mathcal{Z}$:
%
%	\quad \Send (m) $\rightarrow \mathcal{A}_{\mathcal{D}}$
%
%\OnInput \inmsg{activates}{$p_j$} from $\F_{\msf{Bracha}}$:
%
%	\quad \If first message in round $r$:
%
%		\quad \quad Deliver messages from $\Fsync{p_j}{p_i}$ to $p_i$ through $(\msf{fetch})$ and simulate state changes.
%
%\vspace{2mm} \hrule \vspace{2mm}
%
%When protocol terminates, obtain output value $v$. Deliver $v \rightarrow \F_{\msf{Bracha}}$ as the dealer $\mathcal{D}$.
%
%\end{bbox}
\begin{bbox}[title={Simulator $S_{\msf{Bracha}}$}]

Simulate real-world parties $\overline{\mathcal{P}} = p_1,...,p_n$ and $\Fsync{p_i}{p_j}, \forall p_i,p_j \in \overline{\mathcal{P}}$

Simulate instance $\overline{\F}$ of $\F_{\msf{clock}}$.

Designate same dealer $\overline{\mathcal{D}}$ as environment.

Simulate dummy adversray $\mathcal{A}_{\mathcal{D}}$

\vspace{2mm} \hrule \vspace{2mm}

Case \#1 ( Dishonest $\mathcal{D}$ ):

-- \OnInput \inmsg{input}{v} from $\mathcal{Z}$ for $\mathcal{D}$:

	\qquad \Send (input,v) $\rightarrow \mathcal{A}_{\mathcal{D}}$ {\em (Passthrough for corrupted parties in real world)}

-- \OnInput \inmsg{m} from $\mathcal{Z}$:

	\qquad \Send (m) $\rightarrow \mathcal{A}_{\mathcal{D}}$

-- \OnInput \inmsg{activates}{$p_j$} from $\F_{\msf{Bracha}}$:

	\qquad \If first message in round $r$:

		\qquad \quad Deliver messages from $\Fsync{p_j}{p_i}$ to $p_i$ through $(\msf{fetch})$ and simulate state changes.

\vspace{2mm} \hrule \vspace{2mm}

When protocol terminates, obtain output value $v$. Deliver $v \rightarrow \F_{\msf{Bracha}}$ as the dealer $\mathcal{D}$.

\end{bbox}

%\end{figure}
\begin{figure}[!h]
	%\begin{bbox}[title={Wrapper $\mathcal{W}_{\msf{In-O(1)}} (\F)$}]
%
%Initialize $\msf{crnd} := 0$, $\msf{lastcrnd} := -1$, $\msf{runqueue} := []$
%
%\vspace{2mm} \hrule \vspace{2mm}
%
%\OnInput \inmsg{In-O(1)}{codeblock e} from $\F$:
%
%	\quad Add $e$ to $\msf{runqueue}$
%
%	\quad \Leak $e \rightarrow \mathcal{A}$
%
%\OnInput \inmsg{deliver}{idx} from $\mathcal{A}$
%
%	\quad $e \leftarrow \msf{runqueue}[idx]$
%
%	\quad Delete $\msf{runqueue}[idx]$
%
%	\quad {\bf Execute} $e$
%
%\vspace{2mm} \hrule \vspace{2mm}
%
%On every activation:
%
%	\quad $\msf{rnd} \leftarrow \F_{\msf{clock}}.\msf{clockread}$
%
%	\quad \If $\msf{rnd} \neq \msf{crnd}$:
%
%		\quad \quad $\msf{lastcrnd} \leftarrow \msf{crnd}$
%
%		\quad \quad $\msf{crnd} \leftarrow \msf{rnd}$
%
%\end{bbox}

%\begin{bbox}[title={Wrapper $\mathcal{W}_{\msf{O(1)}}$}]
%
%Initialize $\msf{crnd} := 0$, $\msf{lastcrn} := -1$, $\msf{runqueue} := []$
%
%\vspace{2mm} \hrule \vspace{2mm}
%
%\OnInput \inmsg{In O(1)}{codeblock e} from $\F$:
%
%	\quad Add $e$ to $\msf{runqueue}[\msf{crnd}+1]$ 
%
%\OnInput \inmsg{deliver}{idx} from $\mathcal{A}$:
%
%	\quad Pop $e \leftarrow \msf{runqueue}[\msf{crnd}][idx]$
%
%	\quad Execute $e$
%
%\end{bbox}

\begin{bbox}[title={$\Fbc (\mathcal{D}, \mathcal{P} = p_1,...,p_n)$}]

Intialize $x_\dealer := \bot, \ell := 1, \forall p_i : t_i = |\mathcal{P}|$

\vspace{2mm} \hrule \vspace{2mm}

-- \OnInput \inmsg{input}{$v$} from \Partyi:
	
	\qquad Set $x_\dealer := v$

	\qquad \Leak (input,$v$) $\rightarrow \mathcal{A}$

-- \OnInput \inmsg{output} from \Partyi:
	
	\qquad \If ($p_i = \dealer$) and ($x_\dealer$ not set): ignore 

	\qquad \Else \If $(t_i > 0)$: Set $t_i := t_i - 1$

		\qquad \quad \If $(\forall p_i \in \mathcal{H})$: Set $\ell := \ell + 1$

	\qquad \Else \If $(t_i = 0)$ and $(\ell < Rnd)$: \Send (early) $\rightarrow p_i$

	\qquad \Else \If $(y_1,...,y_n)$ not set:

		\qquad \quad Set $y_1,...,y_n := x_\dealer$

\end{bbox}

%\begin{bbox}[title={Simulator $S_{\msf{Bracha}}$}]
%
%Simulate real-world parties $\overline{\mathcal{P}} = p_1,...,p_n$ and $\Fsync{p_i}{p_j}, \forall p_i,p_j \in \overline{\mathcal{P}}$
%
%Simulate instance $\overline{\F}$ of $\F_{\msf{clock}}$.
%
%Designate same dealer $\overline{\mathcal{D}}$ as environment.
%
%Simulate dummy adversray $\mathcal{A}_{\mathcal{D}}$
%
%\vspace{2mm} \hrule \vspace{2mm}
%
%Case \#1 ( Dishonest $\mathcal{D}$ ):
%
%\OnInput \inmsg{input}{v} from $\mathcal{Z}$ for $\mathcal{D}$:
%
%	\quad \Send (input,v) $\rightarrow \mathcal{A}_{\mathcal{D}}$ {\em (Passthrough for corrupted parties in real world)}
%
%\OnInput \inmsg{m} from $\mathcal{Z}$:
%
%	\quad \Send (m) $\rightarrow \mathcal{A}_{\mathcal{D}}$
%
%\OnInput \inmsg{activates}{$p_j$} from $\F_{\msf{Bracha}}$:
%
%	\quad \If first message in round $r$:
%
%		\quad \quad Deliver messages from $\Fsync{p_j}{p_i}$ to $p_i$ through $(\msf{fetch})$ and simulate state changes.
%
%\vspace{2mm} \hrule \vspace{2mm}
%
%When protocol terminates, obtain output value $v$. Deliver $v \rightarrow \F_{\msf{Bracha}}$ as the dealer $\mathcal{D}$.
%
%\end{bbox}
%\begin{bbox}[title={Simualator $S_{\msf{Bracha}}$}]
%
%Simulate real-world parties $\mathcal{\overline{P}} = p_1,..,p_n$ anbd $\Fsync{p_i}{p_j}, \forall p_i,p_j \in \mathcal{P}$ and corrupt $t$ of them.
%
%Simulate instance $\overline{\F}$ of $\F_{\msf{clock}}$ and instance $\overline{\mathcal{W}}$ of wrapper $\mathcal{W}_{O(1)}$.
%
%Designate the same dealer $\overline{\mathcal{D}}$ as the ideal protocol.
%
%Simulate the real world adversary $\mathcal{A}$
%
%\vspace{2mm} \hrule \vspace{2mm}
%
%\OnInput \inmsg{T} from $\F_{\msf{Bracha}}$ \emph{(input to $\F_{\msf{Bracha}}$ from $\overline{\mathcal{D}}$)}:
%
%	\quad Submit $T$ to $\overline{\mathcal{D}}$
%
%	\quad Simulate state changes in all praties until $\overline{\F}.\msf{round}$ increments
%
%\OnInput \inmsg{deliver}{idx} from $\mathcal{Z}$:
%
%	\quad \Send (deliver,idx) $\rightarrow$ $\overline{\mathcal{W}}$
%
%\OnInput \inmsg{clockupdate}{$p_i$} from $\mathcal{Z}$:
%
%	\quad \If $p_i$ is corrupted: \Send (clockupdate) $\rightarrow \overline{\F}$
%
%	
%
%\end{bbox}

\end{figure}
\begin{figure}[!h]
	%\begin{bbox}[title={$\Pi_{\msf{Bracha}} (\mathcal{D}, \mathcal{P} = p_1,...,p_n)$ in $\F_{\msf{sync}}$-hybrid}]
\begin{bbox}[title={$\Pi_{\msf{Bracha}} (\mathcal{D}, \mathcal{P} = p_1,...,p_n)$ in $\F_{\msf{BD-SEC}}$-hybrid}]

Initialize $\msf{BQ} := \frac{\msf{ceil}(n+t)}{2}$, $\msf{init} := crnd$, $\msf{out} := \emptyset$

\vspace{2mm} \hrule \vspace{2mm}

% dealer input INPUT
{\bf Dealer $\mathcal{D}$ Protocol}

-- \OnInput \inmsg{input}{m} from $\mathcal{Z}$:

	\qquad \For $p_i \in \mathcal{P}$:

		\qquad \quad \Send $\msf{VAL}(m) \rightarrow \Fsync{\mathcal{D}}{p_i}$

\vspace{2mm} \hrule \vspace{2mm}

{\bf Party $p_i$ Protocol}

% on input VAL
-- \OnInput \inmsg{$\msf{VAL}(m)$} from $\F_{\msf{sync},\mathcal{D},p_i}$ (once, round $\msf{init}+1$):

	\qquad \For $p_j \in \mathcal{P}$: \Send $\msf{ECHO}(m) \rightarrow \Fsync{p_i}{p_j}$

-- \OnInput \inmsg{$\msf{ECHO}(m)$} from $\Fsync{p_j}{p_i}$ (round $\msf{init}+2$):

	\qquad \If received $\msf{ECHO}(m)$ from $\msf{BQ}$ parties:

		\qquad \quad \For $p_j \in \mathcal{P}$: \Send $\msf{READY}(m) \rightarrow \Fsync{p_i}{p_j}$ 
% on input READY
-- \OnInput \inmsg{$\msf{READY}(m)$} from $\Fsync{p_j}{p_i}$ (round $\msf{init}+3$):

	\qquad \If received $\msf{READY}(m)$ from $2t+1$ parties:

		\qquad \quad $\msf{out} := m$

% on innput OUTPUT 
-- \OnInput \inmsg{output} from $\mathcal{Z}$:

	\qquad \If $\msf{out} \neq \emptyset$: \Output $\msf{out}$ 

	\qquad \Else On $j^{th}$ activation in this round:

		\qquad \quad \Send $(\msf{fetch}) \rightarrow \Fsync{p_j}{p_i}$

		\qquad \quad $m \leftarrow \Fsync{p_j}{p_i}$

\vspace{2mm} \hrule \vspace{2mm}

\If not received $2t + 1$ \msf{READY}(\textunderscore) messages by $\msf{init} + 4$:

	\qquad \Output $\bot$

\end{bbox}


\end{figure}

{\bf Theorem.} {\em Protocol $\Pi_{\msf{Bracha}}$ securely realized $\F_{\msf{Bracha}}$ in the $\{\F_{\msf{BD-SEC}},\F_{\msf{CLOCK}}\}$-hybrid world. Assume a stateic adversary corrupted up to $\frac{n}{3}$ parties.}

Consider the simulator, $\mathcal{S}$, above.

If the dealer $\mathcal{D}$ is honest: In the ideal world, $\mathcal{D}$ gives input $v$ to $\F_{\msf{Bracha}}$ which gives leaks it to $\mathcal{S}$.
The simulator submits the input to it all of the locl $\Fsync{\mathcal{D}}{p_i}$ for $p_i \in  \mathcal{P}$.

$\mathcal{S}$ expects to receive $|\mathcal{P}|$ activations from $\F_{\msf{Bracha}}$ when ideal world parties attempt to read output from the functionality.
In each activation, the simulator sufficiently ensures each party reads messages from all other parties and simualated state changes and increment the local $\overline{\mathcal{F}}_{\msf{clock}}$

The functionality waits $Rnd = 3$ rounds to deliver the output. In the first round $|\mathcal{P}|^2$ activations ensure all \msf{ECHO} messages are sent.
In functionality round 2, activations ensure that all \msf{READY} messages are sent. The final functionality round 3, all \msf{READY}s are delivered and the simulates real world parties all output a value $v$.
By the proof of the Bracha protocol, all real world parties output the same value. The simulator instructs 
\subsection{Extra}

\begin{figure}
	\begin{bbox}[title={\textbf{Functionality} $\F_{\msf{clock}} (\mathcal{P})$}]

Intialize $\forall p_i \in \mathcal{P}: d_i := 0$

\vspace{2mm} \hrule \vspace{2mm}

-- \OnInput \inmsg{RoundOK} from \Partyi:

	\dquad $d_i = 1$

	\dquad \If $\forall p_i \in \mathcal{P}: d_i = 1$:

	\dquad \quad $d_i := 0$ for all $p_i \in \mathcal{P}$

	\dquad \Leak $(\msf{switch},\mathbf{P_i}) \rightarrow \mathcal{A}$

-- \OnInput \inmsg{RequestRound} from \Partyi: 

	\dquad \Send $d_i \rightarrow \mathbf{P_i}$

-- \OnInput \inmsg{corrupt}{$p_i$} from $\mathcal{A}$:

	\dquad Set $\mathcal{H} := \mathcal{H} \cup \{p_i\}$

\end{bbox}


\end{figure}

\begin{figure}
	\begin{bbox}[title={Wrapper $\mathcal{W}_{\msf{Eventually}} (\F)$}]

Initialize $\msf{crnd} := 0$, $\msf{lastcrnd} := -1$, $\msf{runqueue} := []$

\vspace{2mm} \hrule \vspace{2mm}

\OnInput \inmsg{eventually}{codeblock e} from $\F$

	\quad Add $e$ to $\msf{runqueue}$

	\quad \Leak $e \rightarrow \mathcal{A}$

\OnInput \inmsg{deliver}{idx} from $\mathcal{A}$:

	\quad $e \leftarrow \msf{runqueue}[idx]$

	\quad Delete $\msf{runqueue}[idx]$

	\quad {\bf Execute} $e$

\vspace{2mm} \hrule \vspace{2mm}

On every activation:

	\quad $\msf{rnd} \leftarrow \F_{\msf{clock}}.\msf{clockread}$

	\quad \If $\msf{rnd} \neq \msf{crnd}$:

		\quad \quad $\msf{lastcrnd} \leftarrow \msf{crnd}$

		\quad \quad $\msf{crnd} \leftarrow \msf{rnd}$

\end{bbox}

\begin{bbox}[title={$\F_{\msf{3PC}} (\mathcal{D}, \mathcal{P} = p_1,...,p_n, V_C)$}]

Initialize $\msf{buffer} := \emptyset$, $\msf{pending} := False$ %$\msf{flag} := \msf{OK} \in \{\msf{OK},\msf{PENDING}\}$

$quorum := 0$, $d_t := -1$

\vspace{2mm} \hrule \vspace{2mm}

\OnInput \inmsg{input}{T} from $\mathcal{D}$:

	\quad \If $\msf{pending}$: \reject

	\quad $\msf{pending} = True$

	\quad $d_t = \msf{crnd} + 2$

	\quad \For $p_i \in \mathcal{P}$:

		\quad \quad \msf{Eventually} \Send $\msf{ready} \rightarrow p_i$

%\OnInput \inmsg{send}{T} from $\mathcal{D}$:
%
%	\quad \If $\msf{flag} = \msf{PENDING}$: \reject
%
%	\quad $d_t = \msf{crnd} + 2$
%
%	\quad \For $p_i \in \mathcal{P}$:
%
%		\quad \quad \msf{Eventually} \Send $T \rightarrow p_i$
%

\OnInput \inmsg{status}{s} from \Partyi:

	\quad \If not $\msf{pending}$: \ignore

	\quad \If first ``\msf{status}`` by ${\bf P_i}$:

		\quad \quad \If $s = OK$: $ok = ok + 1$

		\quad \quad \If $s = Abort$: $abort = abort + 1$

	\quad \If $ok \geq V_C$:

		\quad \quad \msf{pending} = $False$, $ok,abort = 0$, $d_t = -1$

		\quad \quad \For $p_i \in \mathcal{P}$:
			
			\quad \quad \quad \msf{Eventually} \Send $\msf{commit}{T} \rightarrow p_i$

	\quad \If $abort \geq V_A$:

		\quad \quad \msf{pending} = $False$, $ok,abort = 0$, $d_t = -1$


%\OnInput \inmsg{commit}{T} from \Partyi:
%
%	\quad \If $T \neq \msf{buffer}[-1]$ or $\msf{flag} = \msf{OK}$: \reject
%
%	\quad \If first ``\msf{commit}'' on $T$ by ${\bf P_i}$:
%
%		\quad \quad $quorum = quorum + 1$
%
%	\quad \If $quorum \geq V_C$:
%
%		\quad \quad \msf{flag} = \msf{OK}
%
%		\quad \quad $quroum = 0$, $d_t = -1$
%
%		\quad \quad \For $p_i \in \mathcal{P}$:
%
%			\quad \quad \quad \msf{Eventually} \Send $\msf{commit}(T) \rightarrow p_i$
%

\vspace{2mm} \hrule \vspace{2mm}

On every activation:

	\quad \If \msf{pending} and $\msf{crnd} \geq d_t$:

		\quad \quad Remove last element in \msf{buffer}

		\quad \quad $d_t = -1$, $ok = 0$, $abort = 0$

\end{bbox}

\end{figure}

%\begin{bbox}[title=asd]
%hello
%\end{bbox}


\bibliographystyle{plain}
\emergencystretch 1.5em
\bibliography{bibuccontracts}


\end{document}
