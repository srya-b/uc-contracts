This section will briefly outline the UC model and the synchronous model propose by Katz et al.~\cite{katz-clock}.
Next, we introduce the asynchronous byzantine broadcast protocol proposed in ~\cite{bracha-broadcast} and how it is cast to the synchronous model.

\subsection{UC Framework}
\todo{brief on the UC framework. Key points are adversary model, existence of environment, real/ideal paradigm, composition operation, asynchronous communication}

\subsection{Synchronous Protocols in UC}
As mention above, the UC framework is built to model asynchronous communication, i.e. there are not guarantees on the delay of messages from one party to another.
%More importantly, this means that it is impossible for any party to distinguish between a faulty process and one whose messages haven't been received yet.
More importantly this makes it difficult to reason about liveness and termination properties of protocols as faulty processes are indistinguishable from those whose messages haven't arrived yet.
Naturally, the asynchronous model stronger than a synchronous model as less assumptions are required when proving the security of protocols.
However, in some cases additional properties are required such as guaranteed termination and imput completeness.
In the case of distributed applications, we wish to reason about their liveness properties which are difficult to achieve in asynchronous systems~\cite{flp}.

\paragraph{Model} 
The first step in the synchronous model is to apply bounded delay to the point-to-point channels that parties have between each other.
One of the key reasons termination can not be guarnateed in the asynchronous model is that the adversary can impose ubounded delay and force processes to wait indefinitely and not make any progress.
We use the bounded delay channel proposed in ~\cite{katz-clock}, \Fbdsec, described in Figure~\ref{fig:bdsec}.

\begin{figure}
	\begin{bbox}[title={\textbf{Functionality} $\F_{\msf{BD-SEC}}^{\delta,\ell}(p_s,p_r)$}]

Initialize $M := \bot$ and $D := 1$ and $\hat{D} := 1$

-- \OnInput \inmsg{m} from $\mathbf{p_s}$:

	\dquad Set $D := 1$, $M := m$

	\dquad \Leak ({\bf send},$M$) $\rightarrow \mathcal{A}$

-- \OnInput \inmsg{fetch} from $p_r$:

	\dquad Set $D = D - 1$

	\dquad \If $D = 0$: \Send ({\bf sent},$M$) $\rightarrow p_r$

-- \OnInput \inmsg{delay}{T} from $\mathcal{A}$:

	\qquad \If $\hat{D} + T \leq \delta$: 

		\qquad \quad $D := D + T$, $\hat{D} := \hat{D} + T$

		\qquad \quad \Send ({\bf delay},$T$) $\rightarrow \mathcal{A}$

-- \OnInput \inmsg{replace}{$m'$}{$T'$} from $\mathcal{A}$:

	\qquad \If $p_s$ corrupted, $D > 0$ and $T'$ is valid:

		\qquad \quad $D := T'$ and $M' = m$
\end{bbox}


	\caption{A bound delay channel betwee two parties. Parameterized by bounded delay, $\delta$, and leak function, $\ell$. Without any other crypgoraphy, the most basic leak function is $\ell(m) := m$.}
	\label{fig:bdsec}
\end{figure}

The channel forces the receiver to continuously poll the the channel until a message is available.
An internal count $D$ tracks how many activations are remaining until the message it delivered.
The adversary can delay the message at any time but the total delay imposed, denoted by $\hat{D}$, is bounded by $\delta$.
Howeer, a bounded delay channel alone is not sufficient to achieve input completeness and guaranteed termination.
Without another functionality to synchronize all participants, some party may begin computation and sending message for the next round before other honest parties have completed the current round.
\todo{Add some more exposition about clock functionality}.

The synchronization primitive in ~\cite{katz-clock}, denotes \Fclock, is shown in Figure ~\ref{fig:functionality:clock}.
The functionality maintains an internal but $d_i$ for each party in $\mathcal{P}$. 
In every round, each party performs all necessary actions and indictes it is finished by sending the \msf{RoundOK} signal to \Fclock.
As mentioned previously, the party waits until its bit is reset, i.e. every party has finished its round.
\todo{the fclock functionality in katz only waits until every corrupted party indicates the round is over, is that correct or a type? why (not)?}

\begin{figure}
	\begin{bbox}[title={\textbf{Functionality} $\F_{\msf{clock}} (\mathcal{P})$}]

Intialize $\forall p_i \in \mathcal{P}: d_i := 0$

\vspace{2mm} \hrule \vspace{2mm}

-- \OnInput \inmsg{RoundOK} from \Partyi:

	\dquad $d_i = 1$

	\dquad \If $\forall p_i \in \mathcal{P}: d_i = 1$:

	\dquad \quad $d_i := 0$ for all $p_i \in \mathcal{P}$

	\dquad \Leak $(\msf{switch},\mathbf{P_i}) \rightarrow \mathcal{A}$

-- \OnInput \inmsg{RequestRound} from \Partyi: 

	\dquad \Send $d_i \rightarrow \mathbf{P_i}$

-- \OnInput \inmsg{corrupt}{$p_i$} from $\mathcal{A}$:

	\dquad Set $\mathcal{H} := \mathcal{H} \cup \{p_i\}$

\end{bbox}


	\label{fig:functionality:clock}
\end{figure}

\paragraph{Protocols in the $\{\F_{\msf{BD-SEC}}^{\delta,\ell}, \F_{\msf{clock}}\}$-hybrid model} 
We briefly recap how a protocol operates in this world to achieve the desired properties.
Each protocol begins with its local round set to 1.
In each round the protocol performs some local computation, receives messages from other parties from previous rounds, and potentially sends messages of its own or future rounds from its incoming and outgoing channels, \Fbdsec.
The last thing each itms does in the round is send the \msf{RoundOK} signal to $\F_{\msf{clock}}$.
If it is activated again, it checkes whether the all other parties have also indicated completion of their round.
When it's bit is reset to 0, it increments its local round counter and begins executing the next round.
Otherwise, it doesn't do anything until the round is finished.
